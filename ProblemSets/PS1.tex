
\documentclass{article}
\usepackage[utf8]{inputenc}

\title{PS1 Hoehne}
\author{Jordan Hoehne }
\date{January 2018}

\usepackage{natbib}
\usepackage{graphicx}
\usepackage{amsmath}

\begin{document}

\maketitle

\section{Introduction}
Hello, my name is Jordan Hoehne. Yes, my last name is as confusing as it sounds. I have started prefacing it as 6-characters, H-O-E...H-N-E and it still tends to confuse people. I began coding in the Summer of 2017 and since then I have been spending several hours each week studying code and the sort in order to implement the economic theory I developed in my undergraduate career here at the University of Oklahoma. My favorite project has been my Twitter bot, which I learned about API's and dabbled in automation. I wanted to take this class as a way to reinforce what I have taught myself. I wanted to validate and bring to life what I learned in undergrad as well. For example, calculating elasticity given 5 data points is easy, but the firms that are hiring us are using thousands of records to inform their decision makers. Thus, by taking this class I hope to learn how to efficiently program to meet their expectations. After graduation, I want to join a team of phenomenal individuals that are going to accomplish great things and change the way humans do things. Through data science, I hope to help people live longer, take on less risky activities, and do more with their time here on earth.

\section{Equation}

\begin{equation}
    a^2 + b^2 = c^2
\end{equation}



\centering

\end{document}

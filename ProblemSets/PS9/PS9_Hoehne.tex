\documentclass{article}
\usepackage[utf8]{inputenc}

\title{PS9 Hoehne}
\author{Jordan Hoehne}
\date{April 2018}

\usepackage{natbib}
\usepackage{graphicx}

\begin{document}

\maketitle

\section{PS9}
Question 5\\
Dimensions of housing.train: [1:404] \\
Thus, the training data is a vector of 404 integers.\\
\\
Question 6\\
What is the optimal value of lambda?\\
The optimal value of lambda is 0.00288. \\
What is the in-sample RMSE?\\
The in-sample RMSE is 0.1881683. \\
What is the out-of-sample RMSE (i.e. the RMSE in the test data)?\\
The out-of-sample RMSE is 0.2173507 \\
\\
Question 7\\
What is the optimal value of lambda now?\\
The optimal value of lambda is 0.00314.\\
What is the in-sample RMSE?\\
The in-sample RMSE is 0.1899243.\\
What is the out-of-sample RMSE (i.e. the RMSE in the test data)?\\
The out-of-sample RMSE is 0.1913142.\\
\\
Question 8\\
What are the optimal values of lambda and alpha after doing 6-fold cross validation?\\
The optimal value of lambda is 0.00606. \\
The optimal value of alpha is 0.582. \\
What is the in-sample RMSE?\\
The in-sample RMSE is 0.2008356. \\
What is the out-of-sample RMSE?\\
The out-of-sample RMSE is 0.1913142.\\
Does the optimal value of alpha lead you to believe that you should use LASSO or ridge regression for this prediction task?\\
Given alpha, I would recommend using the lasso method for the prediction task as it's RMSE is lower than the RMSE of ridge regression. \\
\\
Question 9\\
Why you would not be able to estimate a simple linear regression model on the
housing.train dataframe.\\
Because the linear model is so simple, it's not capable of correctly measuring our data. A linear model would lead to a underfit model.
Using the RMSE values of each of the tuned models in the previous three questions, comment on where your model stands in terms of the bias-variance tradeoff.\\
Given our models RMSE's continuously result near .2, we have a highly biased model that exhibits low variance.

\end{document}
